\documentclass[a4paper]{article}
\usepackage[french]{babel}
\usepackage{hyperref}
\usepackage{lmodern,textcomp}
\usepackage{romande}
\usepackage[utf8]{inputenc}
\usepackage[T1]{fontenc}
%\usepackage{parskip}
\usepackage{titlesec}
\usepackage{verbatim}
\usepackage{paralist}
\usepackage{parskip}
\title{La femme de ménage}
\date{}
\author{}
\begin{document}
\maketitle
\begin{center}

[\emph{Mike, vêtu de femme de ménage, traverse la scène. Il ramasse un bout de
papier et le met dans la poubelle. Il quitte la scène et une dramatique
radio des années 20 se met en route, accompagnée de quelques images...}]
\end{center}
La femme de ménage ne le savait pas, mais elle vient de jeter un document
très important.

Assistons donc à la scène où la diligente femme de ménage fouille dans les
déchets pour trouver le papier.

Attention au camion poubelle!

Après cet échec, la femme de ménage voyage en Indochine, ou la main-d’œuvre
n’est pas chère, afin de créer une entreprise capable de reproduire le document
à l’identique.

Ça a bien marché! On a réussi à faire un fac-similé parfait du document.

Sauf que l'on a fait tomber le document et la femme de ménage, toujours
diligente, l’a ramassé et l’a jeté dans la poubelle.

Après avoir refait le même exercice dans plusieurs pays asiatiques, la femme
de ménage s’est livrée à la prostitution.

Ça a bien marché! On a suffisamment d’argent pour embaucher une équipe de chercheurs qui
classifieront toutes les connaissances humaines, dont forcément le document
escompté, dans une grande encyclopédie.

Sauf qu’il leur faudra 1 003 ans pour finir le projet!

Que faire? La femme de ménage
commence à courir, faisant ralentir le temps selon la théorie
de la relativité. Mais au moment de frôler la vitesse de la lumière, la fatigue
commence à s'installer. Epuisée, la femme de ménage décide de se congeler en
attendant que les recherches s'achèvent.

Au bout de 1 003 ans de gel…

on lui demande de bien vouloir patienter 1 003 ans supplémentaires…

au bout des 1 003 ans de gel supplémentaires…

ça a bien marché! Les chercheurs ont construit une encyclopédie à
l'intérieure de laquelle se trouve le document escompté.

Sauf que la femme, dépourvue de son ouïe, sa vue et toute sensation
corporelle à cause de cette longue période de congélation à laquelle se
rajoutent de nombreuses maladies contractées en Asie du Sud, n’en est pas
consciente.

Que faire?

La femme de ménage, toujours futée, s’est elle aussi penchée sur la problématique
du document pendant les 2 006 ans de gel. Ses nombreux calculs ont abouti
à une question qui, une fois résolue, lui redonnerait à la fois l’ouïe, la vue,
les terminaisons nerveuses et le contenu du document. Cette question, qui se
forme dans sa tête depuis vingt siècles, monopolisant son esprit, éclipsant
toute autre pensée, est la question suivante:

%Do you speak English?
\end{document}